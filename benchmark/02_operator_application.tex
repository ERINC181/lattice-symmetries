\documentclass[border=5pt]{standalone}
\usepackage{verbatim}

\usepackage{pgfplots}
\pgfplotsset{compat=1.15}
\usepgfplotslibrary{colorbrewer}

\pgfplotsset{
    cycle list/Pastel2-8,
}
\begin{document}
\begin{tikzpicture}
    \begin{axis}[
        ybar,
        width=9cm, height=5cm,
        ylabel={\small Relative time},
        axis x line*=bottom,
        axis y line=left,
        log origin=infty,
        bar width=0.25cm,
        xmin={[normalized]-0.5},
        legend entries={\texttt{lattice\_symmetries}, \texttt{QuSpin}},
        every axis legend/.append style={
            at={(0.5, -0.3)},
            anchor=north,
            line width=0.7pt,
            legend columns=-1,
            /tikz/every even column/.append style={column sep=0.5cm},
        },
        every axis/.append style={
            line width=1pt,
            tickwidth=0pt,
            tick style={line width=0.8pt},
            tick label style={font=\footnotesize},
            label style={font=\small},
            legend style={font=\small},
        },
        ymin={0.1},
        ytick={5,10,15,20},
        xtick=data,
        x tick label style={rotate=40, anchor=north east},
        every node near coord/.append style={rotate=90, anchor=west, font=\scriptsize},
        % nodes near coords={\pgfmathprintnumber[precision=2]{\pgfplotspointmeta}},
        symbolic x coords={
            $5 \times 5$,
            $5 \times 6$,
            $4 \times 8$,
            $5 \times 7$,
            $6 \times 6$,
            $5 \times 8$
        },
    ]
        \addplot [
            draw=black,
            fill=Pastel2-B,
            nodes near coords={\pgfmathprintnumber[fixed, fixed zerofill, precision=2]{\pgfplotspointmeta}}
        ] table [col sep=tab, meta=system, y expr=1.0, point meta=\thisrow{ls}]{cn71/02_operator_application_1.dat};
        \addplot [
            draw=black,
            fill=Pastel2-C
        ] table [col sep=tab, y expr=\thisrow{quspin}/\thisrow{ls}]{cn71/02_operator_application_1.dat};
    \end{axis}
\end{tikzpicture}
\end{document}
